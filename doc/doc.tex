\documentclass[10pt, a4paper]{article}
\usepackage[utf8]{inputenc}
\usepackage[ngerman]{babel}
\usepackage[parfill]{parskip}
\usepackage[german, affil-it]{authblk}

\usepackage{setspace}
\onehalfspacing

\title{Dokumentation: DigitalHeartMonitor}
\author[1]{Julia Ohlhöft}
\author[1]{Thore Hasselbring}
\author[1]{William Klaffke}
\affil[1]{TH Lübeck}
\date{Stand: \today}


\begin{document}
\maketitle

\tableofcontents

\section{Einleitung}
Dies ist die Dokumentation zur Herstellung und Programmierung eines Herzratenmonitors auf Arduinobasis. Dieser soll als Prototyp für Praktika im BME (Biomedical Engineering) Studiengang dienen.

Das Ziel ist, dass zwei unabhängig von einander arbeitende Teams im Praktikum einen solchen Monitor erstellen. Hierbei soll sich eine Gruppe um die Hardware der Signalverarbeitung und die andere um die Software zur Analyse und Darstellung der Messung kümmern.

\section{Programmteile}

\subsection{\texttt{\#include} - Direktiven}
Das Programm benötigt drei Header-Bibliotheken:
\begin{itemize}
	\item Arduino.h: nur, wenn statt der Arduino-IDE eine andere Entwicklungsumgebung verwendet wird.
	\item Wire.h: mglw. für LiquidCrystal.h notwendig; kann durch \#ifndef-Guards eingeschränkt sein.
	\item LiquidCrystal.h
\end{itemize}	

\subsection{Globale Variablen}
\begin{itemize}
	\item AnalogPin A0: definiert Pin A0 als Eingang.
	\item TimeStep 1000: definiert einen Zeitschritt von 1000 ms pro Durchlauf; kann angepasst werden.
	\item int lowerlimit = 250; int upperlimit = 750: Werte für die obere oder untere Grenze; konservativ, kann angepasst werden.
	\item int highs, lows = 0: Anzahl der Limitüberschreitungen pro Zeiteinhei
\end{itemize}

\subsection{\texttt{void setup()}}
In der setup-Routine wird nur die serielle Schnittstelle und das LCD-Display initialisiert.

\subsection{\texttt{void loop()}}
Die loop-Funktion läuft ständig.

Zunächst wird ein uint32\textunderscore t (vorzeichenloser 32 Bit-Long) \texttt{start} mit der derzeitigen Prozessorzeit millis() definiert. Mit diesem wird in der folgenden Schleife die Laufzeit kontrolliert.

Die Funktion \texttt{setLimits()} legt die obere und untere Grenze der aktuellen Pulsschlagsequenz fest.
Anhand dieser Grenzen wird durch die Funktion \texttt{Diffs()} die Anzahl der oberen und unteren Halbwellen für jeden Schlag gezählt. Die Zählung erfolgt für eine durch eine \texttt{do-while}-Schleife getimte Zeitdauer.

\section{Notizen}
\begin{itemize}
    \item Zählung fehlerhaft, da Rauschen, dass minimal unter Limit fällt als Beat erkannt wird
    
    Lösungsansatz: Array definieren - die niedrigsten Werte ermitteln - deren Indizes ermitteln - Indizes voneinander abziehen - neues Array 
    \item Die Anschlüsse am Sensor gut fixieren, da Lötstellen schnell brüchig werden (bspw. Isoliertape).
    \item Strom bei 5V: 
    \begin{itemize}
        \item Arduino - 25mA
        \item Board - 7mA
        \item Sensor - 12mA
        \item LCD - 1mA
    \end{itemize}
    \item Sensorplatine vor Kurzschluss durch Haut schützen, da sonst ein stark verrauschtes Signal Auftritt. Z.B. die Kontakte mit Isoliertape oder Schrumpfschlauch isolieren.
    \item 
\end{itemize}
    
\end{document}
