\documentclass[10pt, a4paper]{article}
\usepackage[utf8]{inputenc}
\usepackage[ngerman]{babel}
\usepackage[parfill]{parskip}

\title{Dokumentation: DigitalHeartMonitor}
\author{}
\date{Stand: \today}


\begin{document}
\maketitle

\tableofcontents

\section{Einleitung}

\section{Programmteile}

\subsection{\texttt{\#include} - Direktiven}
Das Programm benötigt drei Header-Bibliotheken:
\begin{itemize}
	\item Arduino.h: nur, wenn statt der Arduino-IDE eine andere Entwicklungsumgebung verwendet wird.
	\item Wire.h: mglw. für LiquidCrystal.h notwendig; kann durch \#ifndef-Guards eingeschränkt sein.
	\item LiquidCrystal.h
\end{itemize}	

\subsection{Globale Variablen}
\begin{itemize}
	\item AnalogPin A0: definiert Pin A0 als Eingang.
	\item TimeStep 1000: definiert einen Zeitschritt von 1000 ms pro Durchlauf; kann angepasst werden.
	\item int lowerlimit = 250; int upperlimit = 750: Werte für die obere oder untere Grenze; konservativ, kann angepasst werden.
	\item int highs, lows = 0: Anzahl der Limitüberschreitungen pro Zeiteinhei
\end{itemize}

\subsection{\texttt{void setup()}}
In der setup-Routine wird nur die serielle Schnittstelle und das LCD-Display initialisiert.

\subsection{\texttt{void loop()}}
Die loop-Funktion läuft ständig.

Zunächst wird ein uint32\textunderscore t (vorzeichenloser 32 Bit-Long) \texttt{start} mit der derzeitigen Prozessorzeit millis() definiert. Mit diesem wird in der folgenden Schleife die Laufzeit kontrolliert.

Die Funktion \texttt{setLimits()} legt die obere und untere Grenze der aktuellen Pulsschlagsequenz fest.
Anhand dieser Grenzen wird durch die Funktion \texttt{Diffs()} die Anzahl der oberen und unteren Halbwellen für jeden Schlag gezählt. Die Zählung erfolgt für eine durch eine \texttt{do-while}-Schleife getimte Zeitdauer.

\end{document}
